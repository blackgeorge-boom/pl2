\documentclass[a4paper, 11pt, twoside]{article}
\usepackage{amsmath}
\usepackage{fancyvrb}
\usepackage{listings}% http://ctan.org/pkg/listings
\lstset{
  basicstyle=\ttfamily,
  mathescape
}
\usepackage{array}
\usepackage{bussproofs}
\usepackage[english, greek]{babel}
\usepackage[utf8]{inputenc}
\usepackage{graphicx}
%\usepackage{turnstile}
\usepackage{hyperref}
\newcommand\fnurl[2]{%
  \href{#2}{#1}\footnote{\url{#2}}%
}
\graphicspath{ {screens/} }
\begin{document}
\begin{titlepage}
	\centering
	\includegraphics[width=0.3\textwidth]{ntua}\par\vspace{1cm}
	{\scshape\LARGE Εθνικό Μετσόβιο Πολυτεχνείο \\ Σχολή Ηλεκτρολόγων 
	Μηχανικών και Μηχανικών Υπολογιστών \par}
	\vspace{1cm}
	{\scshape\Large Γλώσσες Προγραμματισμού ΙΙ\par}
	\vspace{1.5cm}
	{\huge\bfseries 4η Σειρά Ασκήσεων \\ \par}
	\vspace{1cm}
	{\Large Συστήματα Τύπων}
	\vspace{2cm}
	
	{\Large\itshape Μαυρογεώργης Νικόλαος \\ 03113087\par }

	\vspace{2cm}
	\vfill

\end{titlepage}

\section*{Μέρος Πρώτο -  Αυτοεφαρμογή}
Έστω, ότι υπάρχει περιβάλλον $\Gamma$ και τύπος $\tau$, έτσι ώστε να ισχύει $\Gamma \vdash xx : \tau$.

Γνωρίζουμε ότι ισχύει ο ακόλουθος κανόνας τύπων:

\begin{otherlanguage}{english}

\[
\begin{tabular}{@{} l >{\centering\arraybackslash}m{.7\textwidth} @{}}
\textsc{Function Application} &
  \begin{prooftree}
  \AxiomC{$\Gamma \vdash e_1 : \tau \rightarrow \tau '$}
  \AxiomC{$\Gamma \vdash e_2 : \tau $}
  \BinaryInfC{$\Gamma \vdash e_1 e_2: \tau '$}
  \end{prooftree}
\\

\end{tabular}
\]
\end{otherlanguage}
o οποίος για $e_1=e_2=x$ γίνεται:

\begin{otherlanguage}{english}

\[
\begin{tabular}{@{} l >{\centering\arraybackslash}m{.7\textwidth} @{}}
\textsc{Function Application} &
  \begin{prooftree}
  \AxiomC{$\Gamma \vdash x : \tau \rightarrow \tau '$}
  \AxiomC{$\Gamma \vdash x : \tau $}
  \BinaryInfC{$\Gamma \vdash x x: \tau '$}
  \end{prooftree}
\\

\end{tabular}
\]
\end{otherlanguage}

Από το λήμμα αντιστροφής (το οποίο αποδεικνύεται άμεσα) για αυτόν τον κανόνα , ισχύει ότι: Αν $\Gamma \vdash x x: \tau$, τότε πρέπει να υπάρχει τύπος $aux$, τέτοιος ώστε $x : aux \rightarrow \tau$ , αλλά και $x : aux$. Προφανώς, οι τύποι $aux$ και $\tau \rightarrow aux$, δεν μπορεί να ταυτίζονται, επομένως καταλήξαμε σε άτοπο.

Άρα, δεν υπάρχει περιβάλλον $\Gamma$ και τύπος $\tau$, έτσι ώστε να ισχύει $\Gamma \vdash xx : \tau$.

\section*{Μέρος Δεύτερο - Αναφορές και Αναδρομή}
Για διευκόλυνση έχουμε συμπεριλάβει και τα \textlatin{let bindings} στη γλώσσα μας.

1. Ένα πρόγραμμα που δεν τερματίζει, θα μπορούσε να είναι η κλήση μιας αναδρομικής συνάρτησης που καλεί τον εαυτό της επ' άπειρον. Η συνάρτηση είναι η:

\begin{otherlanguage}{english}
\begin{lstlisting}
endless $\equiv$ $\lambda$.n : Nat.
    let 
        r = ref ($\lambda$x : Nat.x + 1)
    in
        let
            f = $\lambda$n : Nat.(!r)n
        in
            r := f;
            f n
\end{lstlisting}
\end{otherlanguage}

Οπότε, η κλήση $(endless~1)$, είναι ένα πρόγραμμα που δεν τερματίζει.
\newpage
2. Η συνάρτηση \textlatin{fact} για τον υπολογισμό του παραγοντικού, είναι η εξής:

\begin{otherlanguage}{english}
\begin{lstlisting}
fact $\equiv$ $\lambda$ .n : Nat.
    let 
        r = ref ($\lambda$ x : Nat.x + 1)
    in
        let
            f = $\lambda$ n : Nat.if n <= 1 then 1 else n * ((!r)(n - 1))
        in
            r := f;
            f n 
\end{lstlisting}
\end{otherlanguage}
\end{document}